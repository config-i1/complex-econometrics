% Options for packages loaded elsewhere
\PassOptionsToPackage{unicode}{hyperref}
\PassOptionsToPackage{hyphens}{url}
%
\documentclass[
]{book}
\usepackage{amsmath,amssymb}
\usepackage{lmodern}
\usepackage{iftex}
\ifPDFTeX
  \usepackage[T1]{fontenc}
  \usepackage[utf8]{inputenc}
  \usepackage{textcomp} % provide euro and other symbols
\else % if luatex or xetex
  \usepackage{unicode-math}
  \defaultfontfeatures{Scale=MatchLowercase}
  \defaultfontfeatures[\rmfamily]{Ligatures=TeX,Scale=1}
\fi
% Use upquote if available, for straight quotes in verbatim environments
\IfFileExists{upquote.sty}{\usepackage{upquote}}{}
\IfFileExists{microtype.sty}{% use microtype if available
  \usepackage[]{microtype}
  \UseMicrotypeSet[protrusion]{basicmath} % disable protrusion for tt fonts
}{}
\makeatletter
\@ifundefined{KOMAClassName}{% if non-KOMA class
  \IfFileExists{parskip.sty}{%
    \usepackage{parskip}
  }{% else
    \setlength{\parindent}{0pt}
    \setlength{\parskip}{6pt plus 2pt minus 1pt}}
}{% if KOMA class
  \KOMAoptions{parskip=half}}
\makeatother
\usepackage{xcolor}
\usepackage{color}
\usepackage{fancyvrb}
\newcommand{\VerbBar}{|}
\newcommand{\VERB}{\Verb[commandchars=\\\{\}]}
\DefineVerbatimEnvironment{Highlighting}{Verbatim}{commandchars=\\\{\}}
% Add ',fontsize=\small' for more characters per line
\usepackage{framed}
\definecolor{shadecolor}{RGB}{248,248,248}
\newenvironment{Shaded}{\begin{snugshade}}{\end{snugshade}}
\newcommand{\AlertTok}[1]{\textcolor[rgb]{0.94,0.16,0.16}{#1}}
\newcommand{\AnnotationTok}[1]{\textcolor[rgb]{0.56,0.35,0.01}{\textbf{\textit{#1}}}}
\newcommand{\AttributeTok}[1]{\textcolor[rgb]{0.77,0.63,0.00}{#1}}
\newcommand{\BaseNTok}[1]{\textcolor[rgb]{0.00,0.00,0.81}{#1}}
\newcommand{\BuiltInTok}[1]{#1}
\newcommand{\CharTok}[1]{\textcolor[rgb]{0.31,0.60,0.02}{#1}}
\newcommand{\CommentTok}[1]{\textcolor[rgb]{0.56,0.35,0.01}{\textit{#1}}}
\newcommand{\CommentVarTok}[1]{\textcolor[rgb]{0.56,0.35,0.01}{\textbf{\textit{#1}}}}
\newcommand{\ConstantTok}[1]{\textcolor[rgb]{0.00,0.00,0.00}{#1}}
\newcommand{\ControlFlowTok}[1]{\textcolor[rgb]{0.13,0.29,0.53}{\textbf{#1}}}
\newcommand{\DataTypeTok}[1]{\textcolor[rgb]{0.13,0.29,0.53}{#1}}
\newcommand{\DecValTok}[1]{\textcolor[rgb]{0.00,0.00,0.81}{#1}}
\newcommand{\DocumentationTok}[1]{\textcolor[rgb]{0.56,0.35,0.01}{\textbf{\textit{#1}}}}
\newcommand{\ErrorTok}[1]{\textcolor[rgb]{0.64,0.00,0.00}{\textbf{#1}}}
\newcommand{\ExtensionTok}[1]{#1}
\newcommand{\FloatTok}[1]{\textcolor[rgb]{0.00,0.00,0.81}{#1}}
\newcommand{\FunctionTok}[1]{\textcolor[rgb]{0.00,0.00,0.00}{#1}}
\newcommand{\ImportTok}[1]{#1}
\newcommand{\InformationTok}[1]{\textcolor[rgb]{0.56,0.35,0.01}{\textbf{\textit{#1}}}}
\newcommand{\KeywordTok}[1]{\textcolor[rgb]{0.13,0.29,0.53}{\textbf{#1}}}
\newcommand{\NormalTok}[1]{#1}
\newcommand{\OperatorTok}[1]{\textcolor[rgb]{0.81,0.36,0.00}{\textbf{#1}}}
\newcommand{\OtherTok}[1]{\textcolor[rgb]{0.56,0.35,0.01}{#1}}
\newcommand{\PreprocessorTok}[1]{\textcolor[rgb]{0.56,0.35,0.01}{\textit{#1}}}
\newcommand{\RegionMarkerTok}[1]{#1}
\newcommand{\SpecialCharTok}[1]{\textcolor[rgb]{0.00,0.00,0.00}{#1}}
\newcommand{\SpecialStringTok}[1]{\textcolor[rgb]{0.31,0.60,0.02}{#1}}
\newcommand{\StringTok}[1]{\textcolor[rgb]{0.31,0.60,0.02}{#1}}
\newcommand{\VariableTok}[1]{\textcolor[rgb]{0.00,0.00,0.00}{#1}}
\newcommand{\VerbatimStringTok}[1]{\textcolor[rgb]{0.31,0.60,0.02}{#1}}
\newcommand{\WarningTok}[1]{\textcolor[rgb]{0.56,0.35,0.01}{\textbf{\textit{#1}}}}
\usepackage{longtable,booktabs,array}
\usepackage{calc} % for calculating minipage widths
% Correct order of tables after \paragraph or \subparagraph
\usepackage{etoolbox}
\makeatletter
\patchcmd\longtable{\par}{\if@noskipsec\mbox{}\fi\par}{}{}
\makeatother
% Allow footnotes in longtable head/foot
\IfFileExists{footnotehyper.sty}{\usepackage{footnotehyper}}{\usepackage{footnote}}
\makesavenoteenv{longtable}
\usepackage{graphicx}
\makeatletter
\def\maxwidth{\ifdim\Gin@nat@width>\linewidth\linewidth\else\Gin@nat@width\fi}
\def\maxheight{\ifdim\Gin@nat@height>\textheight\textheight\else\Gin@nat@height\fi}
\makeatother
% Scale images if necessary, so that they will not overflow the page
% margins by default, and it is still possible to overwrite the defaults
% using explicit options in \includegraphics[width, height, ...]{}
\setkeys{Gin}{width=\maxwidth,height=\maxheight,keepaspectratio}
% Set default figure placement to htbp
\makeatletter
\def\fps@figure{htbp}
\makeatother
\setlength{\emergencystretch}{3em} % prevent overfull lines
\providecommand{\tightlist}{%
  \setlength{\itemsep}{0pt}\setlength{\parskip}{0pt}}
\setcounter{secnumdepth}{5}
\usepackage{booktabs}
\usepackage{amsthm, graphicx, mathtools, amssymb, amsmath}
\makeatletter
\def\thm@space@setup{%
  \thm@preskip=8pt plus 2pt minus 4pt
  \thm@postskip=\thm@preskip
}
\makeatother

\newtheorem*{adamCitation}{Citation}
\newtheorem*{remark}{Remark}
\newtheorem*{definition}{Definition}
\newtheorem*{task}{Task}
\newtheorem*{solution}{Solution}
\usepackage{booktabs}
\usepackage{longtable}
\usepackage{array}
\usepackage{multirow}
\usepackage{wrapfig}
\usepackage{float}
\usepackage{colortbl}
\usepackage{pdflscape}
\usepackage{tabu}
\usepackage{threeparttable}
\usepackage{threeparttablex}
\usepackage[normalem]{ulem}
\usepackage{makecell}
\usepackage{xcolor}
\ifLuaTeX
  \usepackage{selnolig}  % disable illegal ligatures
\fi
\usepackage[]{natbib}
\bibliographystyle{elsarticle-harv}
\IfFileExists{bookmark.sty}{\usepackage{bookmark}}{\usepackage{hyperref}}
\IfFileExists{xurl.sty}{\usepackage{xurl}}{} % add URL line breaks if available
\urlstyle{same} % disable monospaced font for URLs
\hypersetup{
  pdftitle={Time Series Analysis and Forecasting with Complex Dynamic Models},
  pdfauthor={Sergey Svetunkov and Ivan Svetunkov},
  hidelinks,
  pdfcreator={LaTeX via pandoc}}

\title{Time Series Analysis and Forecasting with Complex Dynamic Models}
\author{Sergey Svetunkov and Ivan Svetunkov}
\date{2022-10-07}

\begin{document}
\maketitle

{
\setcounter{tocdepth}{1}
\tableofcontents
}
\hypertarget{introduction}{%
\chapter*{Introduction}\label{introduction}}
\addcontentsline{toc}{chapter}{Introduction}

The theory of complex variables functions is actively used in a variety of disciplines, including modern physics and engineering sciences. It is relatively easy to describe the complex phenomena that are studied in these areas of science using the models and methods of this branch of mathematics. Social sciences, being much more complicated due to the unpredictability of human behaviour, tend to use simpler instruments for modelling complex processes (references). For example, there are only few scientific publications in which methods and models of the theory of complex variables functions are used in economics, and they typically use that instrument for diagnostics or statistical tests (references) rather than proper modelling.

In 2012, Springer published the monograph ``Complex-valued Modeling in Economics and Finance'' (Svetunkov, 2012), which presented the theory and methodology of modelling using complex variables in economics. Svetunkov (2012) have summarised the main principles of using complex variables in economics and discussed how to estimate some of those models. While, this was the first monograph that discussed the topic, the first paper in this direction was Ben Tamari (1997). The author first introduced Wealth as a complex variable consisting of Output (real part) and Money (imaginary part) and showed how, even with such an elementary representation, interesting new results can be obtained in the area of economics. Unfortunately, this work has gone unnoticed in scientific world. We became aware of this work only in 2016, when Ben Tamari had kindly sent Sergey Svetunkov his paper.
Since 2012, there has been some development in the area of modelling using complex variables, notably a paper by Svetunkov \& Kourentzes (2022) on Complex Exponential Smoothing and Kourentez et al.~(2019) on an error measure based on the idea of complex numbers. Up until now, the modelling with complex variables has been mainly picked up by academics working in the areas of forecasting and engineering. The latter group has been using complex autoregressions for a couple of decades, modelling and predicting signals. The former group has only started using the principles described in Svetunkov (2012). In other disciplines, complex variables are not used directly for model building. The probable reason for this is the lack of the communication between the disciplines and the inherited inertia of academia.

In a try to speed up the adoption of the new instrument, we have written this monograph, summarising the research that has been done in the area of dynamic models since 2012. We should clarify that the term ``dynamic model'' used in this monograph refers to models that have a structure that changes over time. The classical example of such a model is ARIMA (AutoRegression Integrated with Moving Average by Box \& Jenkins, 1976), which allows producing forecasts for a variety of processes based on the existing historical time series. Another example is a multivariate version of ARMA, Vector ARMA or VARMA, which allows modelling the dynamics of several related processes simultaneously. For instance, in business and economics, these models are used for prices, demand forecasting and for capturing complex interactions between macroeconomic indicators over time.
However, all autoregression models that exist today use real variables. In this monograph we will discuss their complex-valued counterparts, namely complex ARIMA and complex VARMA, and study their properties, showing how to identify their orders and how to estimate these models in practice. But before rushing into the discussion, we will explain the basics of random complex variables and complex-valued statistics, which has been developed in signal processing literature, but has not been used to its full potential.

Chapter 1 discusses the theory of random complex variables, conventional and complex-valued statistics, the complex least squares method, maximum likelihood and complex autocorrelation and partial autocorrelation functions. In Chapter 2, we will move to the discussion of simple dynamic model -- complex AR, starting from its properties and slowly moving to its identification and estimation and then to forecasting. After that, in Chapter 3, we will move towards complex MA, again discussing its properties and how one can apply it in practice. The two parts will be united in a cARIMA model in Chapter 4, where we will also discuss seasonal counterpart of the model. Finally, we will move to vector models in Chapter 5, introducing cVAR, cVMA and cVARMA, showing their advantages in comparison with the conventional real valued models. All of this will be supported by examples in R, which will rely heavily on the package ``complex'', developed especially for this monograph. This means that anyone can then use the proposed models for purposes of time series analysis and forecasting.
This work was partially supported by the Russian Foundation for Basic Research (grant No.~19-010-00610\textbackslash19 ``Theory, methods and methodologies of forecasting economic development by autoregression models of complex variables''). Thanks to this support, it became possible to carry out this research in general, forming a new scientific direction in statistical modelling and short-term forecasting.

The authors are grateful to those young scientists of St.~Petersburg Polytechnic University of Peter the Great, who participated in this study, patiently testing the author's hypotheses. They helped forming the theory behind the models discussed in this monograph. We are pleased to mention their names: Evgeny Goltsev, Nikolai Pitukhin, Victoria Matskevich, Yulia Selivanova, Galina Siruk and Nazira Shaikhleeva.

this monograph relies heavily on the \texttt{complex} package for R. Many of the examples in R will use functions from this package, so make sure to install it before running any R code \citep{R-complex}:

\begin{Shaded}
\begin{Highlighting}[]
\FunctionTok{install.packages}\NormalTok{(}\StringTok{"complex"}\NormalTok{)}
\end{Highlighting}
\end{Shaded}

\hypertarget{intro}{%
\chapter{Introduction to random complex variables}\label{intro}}

\hypertarget{theory-of-complex-variables}{%
\section{Theory of complex variables}\label{theory-of-complex-variables}}

This section sets out the basic concepts of the Theory of Complex Variables Functions (TCVF) with some historical overlook of the idea of complex numbers. Hopefully, this will help in understanding the main idea that we will use extensively in the following chapters.

The theory of complex numbers started in 1572 in the small Italian town of Kura. It was in this place and precisely in this year that the manuscript of an Italian mathematician Rafael Bombelli was published. The book was called Algebra. Rafael Bombelli showed in it how to solve the following cubic equation:
\begin{equation}
    x^3 = 15x + 4 .
    \label{eq:BombelliEquation}
\end{equation}
The root of cubic equations at that time was calculated using the formula of Scipione del Ferro. With regards to the task at hand, finding the root should have been carried out as follows:
\begin{equation}
    \begin{aligned}
    x = & \sqrt[3]{\frac{4}{2}+\sqrt{\left(\frac{4}{2}\right)^2 - \left(\frac{15}{3}\right)^3}} + \sqrt[3]{\frac{4}{2}-\sqrt{\left(\frac{4}{2}\right)^2 - \left(\frac{15}{3}\right)^3}} = \\
        & \sqrt[3]{2+\sqrt{-121}} + \sqrt[3]{2-\sqrt{-121}}
    \end{aligned}
    \label{eq:BombelliSolution}
\end{equation}
As can be seen from equation \eqref{eq:BombelliSolution}, there are several square roots, and it follows directly from the right hand side of this equation that to calculate the roots of equation \eqref{eq:BombelliEquation}, we need to extract the square root from the negative number, -121. This means that it will not be possible to find a solution to this equation in the domain of real numbers.

  \bibliography{library.bib,packages.bib,websites.bib}

\end{document}
